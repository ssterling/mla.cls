\documentclass[mla7]{mla}

\title{Sample MLA Document}
\author{John Doe}
\professor{Dr Suzie Que}
\course{\LaTeX\ 101}
\date{\today}

% This will be explained later!
\addbibresource{example.bib}

\begin{document}

\begin{paper}

This is an example document using ``mla.cls''.
The header is automatically printed for you when calling the
``paper'' environment, which is why we jumped straight into
writing our body.

\section{A section}

Our professor prefers we section our paper since it's
\emph{really} long:
\begin{blockquote}
I really hope you do this right, John.
I got tired of reading that twenty-page paper with no logical breaks!
Just divide it into sections already, even though the MLA style guide
doesn't tell you to.
It's technically not \emph{prohibited}, I guess.  \cite{que2019}
\end{blockquote}
That's when the ``\textbackslash{}section'' commands come in handy.
You can even create a subsection!\endnote{It's just like in the
normal ``article'' class.}

\subsection{Subsection}

Notice we cited something above in the blockquote.
That's where ``workscited'' comes in handy with perfect MLA format!
(The ``example.bib'' file follows this whole example.)
But, before that\ldots didn't we specify an endnote somewhere?

\end{paper}

\begin{notes}

% It's also important you use the MLA template
% provided with the class, lest it look wrong.

\printendnotes[mla]

\end{notes}

\begin{workscited}

% It's important to use ``heading=none'' here, since the
% ``workscited'' environment already prints it out for us.

% Also, notice we had to include the bibliography file
% in the preamble.  That's important!

\printbibliography[heading=none]

\end{workscited}

\end{document}
