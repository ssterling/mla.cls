% \iffalse meta-comment
%
% Copyright 2019 Seth Price.
% 
% This file may be distributed and/or modified under the
% conditions of the LaTeX Project Public License, either
% version 1.3 of this license or any later version.
% The latest version of this license is in:
% 
%   https://www.latex-project.org/lppl/lppl-1-3c/
% 
% and version 1.3c or later is part of all distributions
% of LaTeX version 2008/05/04 or later.
%
% \fi

% \iffalse
%
% \section{Identification}
%
% For the \textsf{driver} code:
%
%<*driver>
\ProvidesFile{mla.dtx}
%</driver>
%
% The \Dclass{mla} class has only been tested with \LaTeXe ,
% so we will make sure it doesn't get included in something else:
%    \begin{macrocode}
%<mla>\NeedsTeXFormat{LaTeX2e}
%    \end{macrocode}
% 
% Announce the class name and its version:
%    \begin{macrocode}
%<mla>\ProvidesClass{mla}
%<*mla>
    [2019/05/02 v0.2 MLA Paper Class]
%</mla>
%    \end{macrocode}
%
% \subsection{Documentation driver}
%
% The following is a boilerplate documentation driver for \TeX{}.
% In short, it produces the documentation you're currently reading.
%
%    \begin{macrocode}
%<*driver>
\documentclass{ltxdoc}
\AtBeginDocument{\RecordChanges\CodelineIndex\EnableCrossrefs}
\AtEndDocument{\PrintIndex}

% These bits are some common shorthand I picked up
\newcommand*{\Dfile}[1]{\texttt{#1}}
\newcommand*{\Dopt}[1]{\textsf{\small #1}}
\newcommand*{\Dctr}[1]{\textsl{\small #1}}
\newcommand*{\Dclass}[1]{\textsf{#1}}
\newcommand*{\Dpkg}[1]{\textsf{#1}}
\newcommand*{\Denv}[1]{\textsf{#1}}

% Needed for the appendix
\usepackage{verbatim}

% I just want section headers in my PDF
\usepackage[hidelinks]{hyperref}

% Manually specify PDF metadata
\hypersetup{
    pdfinfo={
        Title=The MLA class,
        Author=Seth Price,
        Creator=LaTeX
    }
}

\GetFileInfo{mla.dtx}

\begin{document}
\DocInput{mla.dtx}
\end{document}
%</driver>
%    \end{macrocode}
% \fi
%
% \CheckSum{0}
%
% \CharacterTable
%  {Upper-case    \A\B\C\D\E\F\G\H\I\J\K\L\M\N\O\P\Q\R\S\T\U\V\W\X\Y\Z
%   Lower-case    \a\b\c\d\e\f\g\h\i\j\k\l\m\n\o\p\q\r\s\t\u\v\w\x\y\z
%   Digits        \0\1\2\3\4\5\6\7\8\9
%   Exclamation   \!     Double quote  \"     Hash (number) \#
%   Dollar        \$     Percent       \%     Ampersand     \&
%   Acute accent  \'     Left paren    \(     Right paren   \)
%   Asterisk      \*     Plus          \+     Comma         \,
%   Minus         \-     Point         \.     Solidus       \/
%   Colon         \:     Semicolon     \;     Less than     \<
%   Equals        \=     Greater than  \>     Question mark \?
%   Commercial at \@     Left bracket  \[     Backslash     \\
%   Right bracket \]     Circumflex    \^     Underscore    \_
%   Grave accent  \`     Left brace    \{     Vertical bar  \|
%   Right brace   \}     Tilde         \~}
%
% \title{The \Dclass{mla} class\thanks{This document corresponds to
%        \Dclass{mla}~\fileversion, dated \filedate.}}
% \author{Seth Price \\ \texttt{ssterling@firemail.cc}}
% \date{\filedate}
%
% \maketitle
%
% \begin{abstract}
% Whilst the \TeX\ family of programs is widely used in the sciences
% and academia overall, there seems to be a lack of support for
% the humanities, which commonly adhere to MLA format for
% student papers and reports.
%
% Though there \emph{are} MLA-style packages available,
% none met the expectations of the author (who was mostly
% just nit-picky about elegance in code).
% So \textit{voil\`a}, there now exists a proper \Dfile{mla.cls}:
% a simple, straightforward class for composing papers
% almost perfectly adherent to the MLA style guide.
% \end{abstract}
%
% \tableofcontents
%
% \StopEventually{} ^^A
%
% \section{Initial code}
% \label{sec:initial_code}
%
% The \Dclass{mla} class is based off the \Dclass{article} class,
% so common macros such as \cs{textit} or \cs{figure} are
% useable as expected.
% Not to say we won't be re-defining a lot of macros, though.
%    \begin{macrocode}
\LoadClass[letterpaper,12pt]{article}
%    \end{macrocode}
%
% \begin{macro}{\mladate}
% MLA requires use of \textit{day month year} dates, not \TeX's
% standard \textit{month, day year}.
% Here we'll define our own \cs{mladate} to give correct output.
%    \begin{macrocode}
\newcommand{\mladate}{%
    \the\day\ 
    \ifcase\the\month
        \or January
        \or February
        \or March
        \or April
        \or May
        \or June
        \or July
        \or September
        \or October
        \or November
        \or December
    \fi
    \the\year
}
%    \end{macrocode}
% \end{macro}
%
% \section{Options}
% \label{sec:options}
%
% Some instructors will still require usage of the seventh edition of MLA.
% MLA 8e is the default for this class, but either can be explicitly specified
% with the \Dopt{mla8} or \Dopt{mla7} options.
% All this actually does is change the edition specified to the
% \Dclass{biblatex} package when loaded in in
% section~\ref{sec:loading_packages}.
%
%    \begin{macrocode}
\DeclareOption{mla7}{%
    \def\blopts{style=mla,noremoteinfo=false,showmedium=true}
}
\DeclareOption{mla8}{%
    \def\blopts{style=mla-new,noremoteinfo=false,showmedium=false}
}
%    \end{macrocode}
%
% \subsection{Processing}
% \label{sec:processing}
%
% The default is to use the eighth edition of MLA.
% Just in case something changes, however, it's wise to explicitly
% specify \Dopt{mla8} in \cs{documentclass}.
%
%    \begin{macrocode}
\ExecuteOptions{mla8}
\ProcessOptions\relax
%    \end{macrocode}
%
% \section{Loading packages}
% \label{sec:loading_packages}
%
% The \Dclass{mla} class requires the following packages to work:
%
%    \begin{macrocode}
\RequirePackage{enotez}
\RequirePackage{fancyhdr}
\RequirePackage{fullpage}
\RequirePackage{indentfirst}
\RequirePackage{ragged2e}
\RequirePackage{times}
\RequirePackage{titlesec}
\RequirePackage{xstring}
%    \end{macrocode}
%
% And after all that, we finally load \Dpkg{biblatex} for the bibliography.
% The other packages here are prerequisites for \Dpkg{biblatex-mla}.
%    \begin{macrocode}
\RequirePackage[american]{babel}
\RequirePackage{csquotes}
\RequirePackage{hanging}
\RequirePackage[hidelinks,pdfusetitle]{hyperref}
\RequirePackage[\blopts,backend=biber]{biblatex}
%    \end{macrocode}
%
% \section{Document Layout}
% \label{sec:document_layout}
% 
% Now for the difficult part.
% A lot of professors are \emph{really} strict when it comes to laying out
% the paper right, so we can't take any chances here.
%
% \subsection{Font}
% \label{sec:font}
%
% The \Dpkg{times} package was already loaded in
% section~\ref{sec:loading_packages}, and the font size was set to
% 12pt when loading the \Dclass{article} class in
% section~\ref{sec:initial_code}.
% This should be metric-compatible with the infamous Times New Roman,
% the \textit{de facto} standard of the MLA format.
%
% \subsection{Line spacing and breaking}
% \label{sec:line_breaking}
%
% The MLA prescribes perfect double-spacing, but \LaTeX\ does it
% a little \emph{too} perfect, putting one less line of text on the
% paper than expected from the ``industry standard'', Microsoft Word.
% We'll compensate by setting the line spacing to \textit{just enough}.
%    \begin{macrocode}
\linespread{1.99}
%    \end{macrocode}
%
% To further adhere to the MLA style guide, we'll forego all hyphenation
% and allow weird line breaks and uncanny whitespace.
%    \begin{macrocode}
\hyphenpenalty 10000
\pretolerance 10000
%    \end{macrocode}
%
% \subsection{Paragraphing}
% \label{sec:paragraphing}
%
% MLA prescribes half-inch indents on the first line of each paragraph
% with no spaces in between.
% (Some professors will request you separate sections of lengthy papers,
% however; just put a paragraph in between the two consisting of ``\cs{par}''.)
%    \begin{macrocode}
\setlength{\parindent}{0.5in}
\setlength{\RaggedRightParindent}{\parindent}
\setlength{\parskip}{0em}
\setlength{\topsep}{0em}
%    \end{macrocode}
%
% MLA also cares more about consistent page count than beautiful typesetting.
% So, we'll accept our fate and allow orphans and widows for the sake of
% uniform line count.
%    \begin{macrocode}
\widowpenalty 0
\clubpenalty 0
\interlinepenalty 0
%    \end{macrocode}
%
% We also want flush-left ragged lines.
% (This is acheived with the \Dpkg{ragged2e} package.)
%    \begin{macrocode}
\RaggedRight
%    \end{macrocode}
%
% \begin{environment}{noindent}
% The \Denv{noindent} environment works a little wonky when working with
% the \Dpkg{ragged2e} package, so we'll re-define it.
%    \begin{macrocode}
\renewenvironment{noindent}{%
    \edef\tmpind{\parindent}
    \setlength{\parindent}{0pt}
}{%
    \setlength{\parindent}{\tmpind}
    \undef{\tmpind}
}
%    \end{macrocode}
%
% \end{environment}
%
% \subsection{Page layout}
% \label{sec:page_layout}
%
% The MLA uses standard ``letter-size'' paper, and that's almost
% all we have in America anyway.
% The paper size was already set to letter when loading the
% \Dclass{article} class in section~\ref{sec:initial_code}.
%
% Furthermore,
% the MLA prescribes uniform one-inch margins on said letter-size paper.
% This is taken care of with the \Dpkg{fullpage} package loaded in
% section~\ref{sec:loading_packages}.  The \cs{textheight} and \cs{textwidth}
% definitions are here just for good measure.
%    \begin{macrocode}
\setlength{\textheight}{9in}
\setlength{\textwidth}{6.5in}
%    \end{macrocode}
%
% \subsection{Running head}
% \label{sec:running_head}
%
% The running head in MLA style is simply the author's surname
% followed by the page number, right-aligned.
% This is managed using the \Dpkg{fancyhdr} and \Dpkg{xstring} packages.
%
%    \begin{macrocode}
\fancypagestyle{norule}{%
    \renewcommand{\headrulewidth}{0pt}
    \renewcommand{\footrulewidth}{0pt}
}
\fancyhf{}
\pagestyle{headings}
\pagestyle{norule}
\fancyhead[RO]{{\StrBehind{\@author}{ }[\last]\last} \thepage}
%    \end{macrocode}
%
% I have to fully admit I don't know what the following code
% actually does, but I tweaked around the variables to get it all
% in the right place.
%
%    \begin{macrocode}
\setlength{\headheight}{18pt}
\setlength{\headsep}{12pt}
\setlength{\voffset}{-34pt}
%    \end{macrocode}
%
% \section{Document Markup}
% \label{sec:document_markup}
%
% \subsection{The header}
% \label{sec:the_header}
%
% \begin{macro}{\title}
% \begin{macro}{\author}
% \begin{macro}{\date}
% The \cs{title}, \cs{author} and
% \cs{date} macros are already defined in \Dfile{latex.dtx}.
% Their definitions are shown here for reference.
%    \begin{macrocode}
% \newcommand*{\title}[1]{\gdef\@title{#1}}
% \newcommand*{\author}[1]{\gdef\@author{#1}}
% \newcommand*{\date}[1]{\gdef\@date{#1}}
%    \end{macrocode}
% \end{macro}
% \end{macro}
% \end{macro}
%
% The following, however, are unique to the \Dclass{mla} class.
%
% \begin{macro}{\professor}
% The name of whoever assigned the paper, i.e. ``Dr Marjorie Stewart''.
%    \begin{macrocode}
\newcommand*{\professor}[1]{\gdef\@professor{#1}}
%    \end{macrocode}
% \end{macro}
% \begin{macro}{\course}
% The course for which this paper was assigned, i.e. ``ENGL 101-02''.
%    \begin{macrocode}
\newcommand*{\course}[1]{\gdef\@course{#1}}
%    \end{macrocode}
% \end{macro}
%
% We make sure the internal macros used to store the above information
% are empty, lest something weird happen.  \cs{date}, however,
% is generally set to the current date.
%    \begin{macrocode}
\title{}
\author{}
\professor{}
\course{}
\date{\today}
%    \end{macrocode}
%
% \begin{macro}{\makemlaheader}
% This command prints out the standard five-line MLA header,
% including the centered title.
% (Note the use of \cs{mladate}; see section \ref{sec:initial_code}.)
%    \begin{macrocode}
\newcommand{\makemlaheader}{%
    \begin{noindent}
        \@author \\
        \@professor \\
        \@course \\
        \mladate \\
        \begin{center}\@title\end{center}
    \end{noindent}
}
%    \end{macrocode}
% \end{macro}
%
% \begin{macro}{\maketitle}
% And we'll just re-define \cs{maketitle} to execute \cs{makemlaheader}.
%    \begin{macrocode}
\renewcommand{\maketitle}{\makemlaheader}
%    \end{macrocode}
% \end{macro}
%
% \subsection{Sectioning}
% \label{sec:sectioning}
%
% \begin{macro}{\section}
% \begin{macro}{\subsection}
% \begin{macro}{\subsubsection}
% Customary section headers are rather straightforward and only consist of
% the section number in Arabic numerals, a space and the section name,
% flush left.
% However, for the sake of clarity, we'll put ours in small-caps.
%
%    \begin{macrocode}
\renewcommand{\thesection}{\@arabic\c@section}
\renewcommand{\thesubsection}{\thesection.\@arabic\c@subsection}
\renewcommand{\thesubsubsection}{\thesubsection.\@arabic\c@subsubsection}
%    \end{macrocode}
%
% Un-fancifying the section headers is acheived using the
% \Dpkg{titlesec} package.
%
%    \begin{macrocode}
\titleformat*{\section}{\normalsize\sc}
\titleformat*{\subsection}{\normalsize\sc}
\titleformat*{\subsubsection}{\normalsize\sc}
\titlespacing*{\section}{0pt}{0pt}{0pt}
\titlespacing*{\subsection}{0pt}{0pt}{0pt}
\titlespacing*{\subsubsection}{0pt}{0pt}{0pt}
\titlelabel{\thetitle. }
%    \end{macrocode}
% \end{macro}
% \end{macro}
% \end{macro}
%
% \TeX\ generally doesn't indent the first paragraph after a
% section header.
% Since this violates MLA style, we'll simply rely on the
% \Dpkg{indentfirst} package as loaded in
% section~\ref{sec:loading_packages}.
%
% \subsection{Block quotation}
% \label{sec:block_quotation}
%
% \begin{environment}{blockquote}
% MLA dictates block quotes be set flush $0.5$in from the left margin.
% We'll re-define the \Denv{blockquote} environment for this purpose.
% The following looks like a complete hack\ldots because it is, honestly.
%
%    \begin{macrocode}
\renewenvironment{blockquote}{%
    \list{}{\leftmargin 0.5in}
    \item[]
    \setlength{\parindent}{0.5in}
    \vspace{-\topsep}
}{%
    \endlist
    \vspace{-\topsep}
}
%    \end{macrocode}
% \end{environment}
%
% \subsection{Paper sections}
%
% \subsubsection{Paper}
%
% \begin{environment}{paper}
% The main content; the body.
% Most everything you write will be in here.
%    \begin{macrocode}
\newenvironment{paper}{%
    \makemlaheader
}{%
    \newpage
}
%    \end{macrocode}
% \end{environment}
%
% \subsubsection{Endnotes}
% \label{sec:endnotes}
%
% \begin{environment}{notes}
% If you have any endnotes, they can be put in the
% \Denv{notes} environment.
%    \begin{macrocode}
\newenvironment{notes}{%
    \begin{noindent}
        \pdfbookmark[0]{Notes}{notes}
        \begin{center}Notes\end{center}
    \end{noindent}
    \vspace{-16pt} % XXX to counter unexplained space
}{%
    \newpage
}
%    \end{macrocode}
% \end{environment}
%
% The MLA doesn't prescribe a strict set of rules for endnotes,
% so the following is just to clean it up and make it consistent.
% We're assuming the required \Dpkg{enotez} package is used;
% refer to its documetation for an explanation of the following.
%    \begin{macrocode}
\setenotez{list-name={}}
\DeclareInstance{enotez-list}{mla}{list}{%
    heading = {},
    format = \normalsize\normalfont,
    list-type = description
}
%    \end{macrocode}
%
% \subsubsection{Bibliography}
% \label{sec:bibliography}
%
% \begin{environment}{workscited}
% This environment creates the bibliography, starting on
% the next counted page number.
%    \begin{macrocode}
\newenvironment{workscited}{%
    \begin{noindent}
        \pdfbookmark[0]{Works Cited}{workscited}
        \begin{center}Works Cited\end{center}
    \end{noindent}
    \vspace{-16pt} % XXX to counter unexplained space
}{%
    \newpage
}
%    \end{macrocode}
% \end{environment}
%
% MLA prescribes a half-inch hanging indent on all
% bibliography entries.
% This is acheived by setting the \cs{bibhang} macro
% defined in the \Dpkg{biblatex} package.
%    \begin{macrocode}
\setlength{\bibhang}{\parindent}
%    \end{macrocode}
%
% \appendix
% \section{Example usage}
% \label{sec:example_usage}
%
% Following is a basic \LaTeXe\ document using \Dfile{mla.cls}
% and its corresponding bibliography.
%
% \subsection{\texttt{mla-example.tex}}
%
% \verbatiminput{mla-example.tex}
%
% \subsection{\texttt{mla-example.bib}}
%
% \verbatiminput{mla-example.bib}
%
% \Finale
\endinput
